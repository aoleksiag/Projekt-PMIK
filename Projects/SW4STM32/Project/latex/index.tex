Project deal with electronic lock. Lock can be open by two way. First approach use 4x4 keypad, user can enter six-\/digit key and confirm it by press \# character on keyboard. In case of enter less than six number, system will ignore it after two second. Character C on keypad removes all digits in code. When lock can be close by $\ast$ button. Second approach use bluetooth module, user send special formed frame to communicate with microcontroller. six digit will be treat as code frame with d on first place change data in rtc clock (\char`\"{}d01\+:12\char`\"{}\}, frame with t change time ("t12\+:10). System stores log in circular buffer, if bluetooth send \$ char log will be return. Bluetooth user can change password.

\begin{DoxyParagraph}{Default pinout L\+CD}

\end{DoxyParagraph}
\begin{DoxyVerb}LCD   GPIO              DESCRIPTION
RS    PB2               Register select, can be overwritten in your project's defines.h file
E     PB7               Enable pin, can be overwritten in your project's defines.h file
D4    PC12              Data 4, can be overwritten in your project's defines.h file
D5    PC13              Data 5, can be overwritten in your project's defines.h file
D6    PB12              Data 6, can be overwritten in your project's defines.h file
D7    PB13              Data 7, can be overwritten in your project's defines.h file\end{DoxyVerb}


\begin{DoxyParagraph}{Default pinout Keypad}

\end{DoxyParagraph}
\begin{DoxyVerb}KeyPad   GPIO             DESCRIPTION
P0       PB3              Input PIN
P1       PA10             Input PIN
P2       PB13             Input PIN
P3       PC4              Input PIN
P4       PA8              Output PIN
P5       PB10             Output PIN
P6       PB4              Output PIN
P7       PC4              Output PIN\end{DoxyVerb}


\begin{DoxyParagraph}{Default pinout U\+A\+RT}

\end{DoxyParagraph}
\begin{DoxyVerb}UART      GPIO          DESCRIPTION
UART_tx   PA10           tx pin
UART_rx   PB7            rx pin\end{DoxyVerb}
 